\documentclass{article}

\usepackage{listings}

\lstset{
basicstyle=\ttfamily,
columns=fullflexible,
frame=single,
breaklines=true,
%postbreak=\mbox{\textcolor{red}{$\hookrightarrow$}\space},
}

\begin{document}


Problem 1

\begin{lstlisting}[language=Python]
def percentage(amount,total):
    percent = amount * 100 // total
    return str(percent) + "%"
\end{lstlisting}

\bigskip
Problem 2

\begin{lstlisting}[language=Python]
def taxi_cost(miles):
    if miles > 10:
        return 4*miles
    else:
    return 3 + 2*miles
\end{lstlisting}

\bigskip
Problem 3

\begin{lstlisting}[language=Python]
def script():
    got_it = False
    while not got_it:
        number = int(input("Enter a three-digit number: "))
        
        if number < 0:
            print("That number is negative.")
        elif number < 10:
            print("That number has only one digit.")
        elif number < 100:
            print("That number has only two digits.")
        elif number >= 1000:
            # The instructions don't say excatly what to do here. 
            print("That number has too many digits!")
        else:
            got_it = True
            
        if not got_it:
            print("Please try again.")

    # Assumes it is 100 and above, but less than 1000.
    h = number // 100
    t = number // 10 % 10
    o = number % 10
    print("Its hundreds digit is "+str(h)+".")
    print("Its tens digit is "+str(t)+".")
    print("Its ones digit is "+str(o)+".")
\end{lstlisting}


\newpage
Problem 4 (a)

\begin{lstlisting}[language=Python]
def squares_of(xs):
    sqrs = []
    i = 0
    while i < len(xs):
        sqrs.append(xs[i]**2)
        i += 1
    return sqrs
\end{lstlisting}

\bigskip
Problem 4 (b)

\begin{lstlisting}[language=Python]
def square_all_of(xs):
    i = 0
    while i < len(xs):
        xs[i] = xs[i] * xs[i]
        i += 1
\end{lstlisting}

\bigskip
Problem 5

\begin{lstlisting}[language=Python]
def describe_list(xs):
    if len(xs) == 0:
        return "That list is empty."
    elif len(xs) == 1:
        return "That list holds the value " + str(xs[0]) + "."
    elif len(xs) == 2:
        s = "That list holds the value " + str(xs[0])
        s += " followed by " + str(xs[1]) + "."
        return s
    else:
        s = "That list holds the sequence "
        i = 0
        while i < len(xs) - 1:
            s += str(xs[i]) + ", "
            i += 1
        s += "and "+str(xs[i])+"."
        return s
\end{lstlisting}


\newpage
Problem 6

\begin{lstlisting}[language=Python]
def number_pyramid(height):
    count = 0
    row = 1
    pyramid = ""
    while row <= height:
        pyramid += " " * (height-row)
        column = 1
        while column <= row:
            pyramid += str(count) + " "
            column += 1
            count = (count + 1) % 10
        if row < height:
            pyramid += "\n"
        row += 1
    print(pyramid)
\end{lstlisting}

\end{document}
